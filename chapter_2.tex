\chapter{%
深層学習による配管6D姿勢推定}

従来のアイソメ図取得にはLIDARセンサーにより3次元点群を取得し図面を作成していたが、
センサーが高価であるというデメリットを抱えていた。そのため、本研究ではLIDARセンサーよりも
比較的安価なRGB-Dカメラを用いてデータセット収集から深層学習やアイソメ図作成までの流れを紹介する。
第2.1節ではRGB-Dカメラを用いた深層学習による配管6D姿勢推定の手順を述べる。
第2.2節では物体認識のネットワーク設計を詳しく説明する。


\section{配管6D姿勢推定方法}
RGB-Dカメラを用いた深層学習による配管のアイソメ図作成の手順はデータ収集、物体認識、姿勢推定、アイソメ図作成の4つに分けられる。
図2.1にそのプロセスの流れを示す。

\begin{figure}[htbt]
	\centering
	 \includegraphics[height=65mm]{system.eps}
	 \caption{RGB-Dカメラを用いた深層学習によるアイソメ図作成方法}
	 \label{fig:f2}
\end{figure}



\section{各章の書き方}

各章(節,および項)においては,
その最初の段落で,
その章または節でなにをしようとしているのかを手短に記述すること.

本文は適当な長さの段落に分け,
段落の最初の一文でその段落の趣旨を明らかにしておく.
その趣旨をまとめると,
その章(節)全体の論理的な筋書きがわかるようになっていることが望ましい.

また,
各章の終わりにも,「考察」または「まとめ」の節を置き,
その章での成果をまとめておく.
最初の段落で「しようとしていること」を述べよ,としたが,
それに対する「出来たのか,出来なかったのか」という答えを,
その章の理論的考察,シミュレーション,
実験結果から論理的に導きだして述べておくこと.
