\chapter{%
深層学習による配管6D姿勢推定}

従来のアイソメ図取得にはLIDARセンサーにより3次元点群を取得し図面を作成していたが、
センサーが高価であるというデメリットを抱えていた。そのため、本研究ではLIDARセンサーよりも
比較的安価なRGB-Dカメラを用いてデータセット収集から深層学習やアイソメ図作成までの流れを紹介する。
第2.1節ではRGB-Dカメラを用いた深層学習による配管6D姿勢推定の手順を述べる。
第2.2節では物体認識のネットワーク設計を詳しく説明する。


\section{配管6D姿勢推定方法}


\subsection{表題}

「表題」は,最も読まれる可能性が高い部分である.
基本的には,その論文で何をし,どういうオリジナリティがあるのか,を簡潔に述べる部分である.
自分のしたことを網羅したいがために,あまり一般的な表題になり過ぎるとかえって焦点がぼやけてしまう.

また,表題の改行位置に気を配らない学生が多いが,
言葉の切れ目で改行するようにしないと不格好である.


\subsection{摘要}

「摘要(概要,アブストラクト)」は,
読者がその論文を読む価値があるかどうか判断するための部分である.
論文の本体と切り離しても,
\begin{enumerate}
	\item 取り組んだ問題
	\item 着眼点
	\item 研究対象
	\item 研究手法
	\item 研究結果
	\item 結論
\end{enumerate}
がわかるように書く.
摘要は,できるだけ簡潔にすることが重要である.
「なぜこの研究をしたのか」という背景説明は不要.
「何をどうして,どういう結果を得たか」
「この論文を読むと読者にとって何が嬉しいのか」を主張する.
「同じ内容を伝えることができるのであれば文章は短い方が偉い」と心掛けること.


\subsection{謝辞}

以下のような人(組織)に対して謝辞を述べるべき \Cite{02} である.
\begin{enumerate}
	\item 研究のアイデアを練る上で,有益な助言をしてくれた人
	\item 実験・調査・観察をする上で有益な助言や技術供与をしてくれた人
	\item 結果の解釈や考察について有益な助言(議論)をしてくれた人
	\item 実験・調査・観察を手伝ってくれた人
	\item 原稿を読んでくれた人
	\item 研究費を出してくれた組織
\end{enumerate}

修士論文,卒業論文には,通常共著者を書かないため,
本来なら共著者に当然入るべき人を含めることができない.
したがって,そのような人がこの論文に対してどのように貢献してくれたのかを,
謝辞で「具体的に」述べて,感謝の言葉とすること.
単に「だれそれに感謝する」の羅列では謝辞にならない.

例えば本手引の謝辞は,論文の謝辞としては具体性に欠け,不十分である.
学生諸君が論文を書く場合には,より具体的に記述することが求められる.


\subsection{緒言}

「緒言」では,
研究の背景と意義,
これまでの関連研究の紹介とその問題点,
研究の目的,
論文の構成,
などを記述する.
要するに「どんな研究をするのか」
「なぜその研究をしなければならないのか」を明確にすることが求められる.

緒言は,
論文で発表しようとしている自らのオリジナリティの工学的な価値を理解してもらえるように書く.
そのためには,研究の背景説明でそれなりの数の文献をサーベイしておくことが不可欠である.
先人の仕事を見ていないのに内容がオリジナルだと言われても信用できない.

有本・川村・伊坂研の過去の修士論文,卒業論文では,なぜか伝統的に,
人類の創始から現代までの壮大な議論や全地球規模の経済問題等を長々と述べている緒言が散見されるが,
工学系の学術論文としては,そのような記述はかなりピント外れで恥ずかしい.


\subsection{本論}

「本論」では,いよいよ研究内容を記述する.
研究内容は,すべて筋道立てて記述すること.

例えば,ある実験をしたのなら,
まずその実験の目的を述べ,実験条件を述べ,実験結果を述べ,
実験結果を考察し,
考察から導かれる結論を述べる.
この結論は,論文全体の結論の中で明確に位置付けられるものなければならない.
目的も位置付けも良く分からない実験データの羅列を読まされることは苦痛である.


\subsection{結言}

「結言」では,
本論で得られた主要な結果を再度まとめて言う.
結言は,摘要の次に読まれる可能性が高い部分であり,
くだらない結論であれば,論文全体がくだらないと見なされる.
自らの成果をきちんと主張すること.

また,今後の課題もここで述べる.


\subsection{参考文献}

原則として,参考文献には公式に発表されたものしか挙げてはいけない.
よく引用されがちなものに学位論文(博士論文,修士論文,卒業論文)やウエブサイトがあるが,
学位論文は学位を申請するために学位審査機関に提出するものであって世界に向けて発表されるものではない.
また,ウエブサイトは一時的な情報であり,いつ削除,更新されるか分からない.
これらについては,どうしてもやむを得ない場合以外の引用は慎むこと.

参考文献リストの体裁については次章に詳しく述べているので参照のこと.


\section{各章の書き方}

各章(節,および項)においては,
その最初の段落で,
その章または節でなにをしようとしているのかを手短に記述すること.

本文は適当な長さの段落に分け,
段落の最初の一文でその段落の趣旨を明らかにしておく.
その趣旨をまとめると,
その章(節)全体の論理的な筋書きがわかるようになっていることが望ましい.

また,
各章の終わりにも,「考察」または「まとめ」の節を置き,
その章での成果をまとめておく.
最初の段落で「しようとしていること」を述べよ,としたが,
それに対する「出来たのか,出来なかったのか」という答えを,
その章の理論的考察,シミュレーション,
実験結果から論理的に導きだして述べておくこと.
