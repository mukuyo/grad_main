\chapter{アイソメトリック図生成}
本章では、6D姿勢推定の結果を基に、配管設備のアイソメ図を生成する手法について述べる。\\
 本章では、以下の構成で解説を進める。
第3.1節では、アイソメ図生成の全体構造について説明する。
第3.2節では、配管接続部のペアマッチングのアルゴリズムについて述べる。
第3.3節では、配管全体の接続構造を探索する方法について説明する。
第3.4節では、最終的に配管情報を描画する方法について述べる。

\section{全体構造}
アイソメ図作成には6D姿勢推定の結果を用いてアイソメ図生成を行う。
図\ref{fig:3-f1}にアイソメ図生成の全体構造を示す。
\begin{figure}[htbt]
  \centering
   \includegraphics[height=55mm]{Figure/iso_flow.eps}
   \caption{アイソメ図生成の流れ}
   \label{fig:3-f1}
\end{figure}

まず、6D姿勢推定の結果を基に、各配管の接続部同士の接続関係を推定するために配管ペアマッチングを行う。
次に、配管全体の接続構造を探索し、最終的に配管情報を描画する。

\section{配管ペアマッチング}
アイソメ図作成では、向かい合った配管同士を繋ぎ合わせるため、接続部のペアを特定する必要がある。
配管ペアマッチングは、6D 姿勢推定によって取得された各接続部の位置と姿勢情報をもとに、配管間の接続関係を判断する。
まず、6D姿勢推定の結果からT字管と曲管の回転行列 $\mathbf{R}_T, \mathbf{R}_C$ と並進ベクトル $\mathbf{t}_T, \mathbf{t}_C$ がそれぞれ与えられているとする。
T字管には接続先となる出口が3つあり、それぞれの方向ベクトルを以下のように定義する。
前方向ベクトルを $\mathbf{d}_{T,f}$、下方向ベクトルを $\mathbf{d}_{T,d}$、上方向ベクトルを $\mathbf{d}_{T,u}$とする。
一方、曲管には出口が2つあり、前方向ベクトルを $\mathbf{d}_{C,f}$、下方向ベクトルを $\mathbf{d}_{C,d}$ とする。

これらの方向ベクトルは、それぞれの回転行列を用いて式(\ref{eq:3-1})のように求められる。

\begin{equation}
  \begin{aligned}
  \mathbf{d}_{T,f} &= \mathbf{R}_T \cdot \mathbf{e}_f, \quad 
  \mathbf{d}_{T,u} = \mathbf{R}_T \cdot \mathbf{e}_u, \quad 
  \mathbf{d}_{T,d} = \mathbf{R}_T \cdot \mathbf{e}_d, \\
  \mathbf{d}_{C,f} &= \mathbf{R}_C \cdot \mathbf{e}_f, \quad 
  \mathbf{d}_{C,u} = \mathbf{R}_C \cdot \mathbf{e}_u
  \end{aligned}
  \label{eq:3-1}
\end{equation}

ここで、$\mathbf{e}_f, \mathbf{e}_u, \mathbf{e}_d$ は基準となる単位ベクトルであり、それぞれ $\mathbf{e}_f = [1, 0, 0]^\mathrm{T}$、$\mathbf{e}_u = [0, 1, 0]^\mathrm{T}$、$\mathbf{e}_d = [0, -1, 0]^\mathrm{T}$ とする。
各方向ベクトルは図\ref{fig:3-f2}に例として示した。
\begin{figure}[htbt]
  \centering
   \includegraphics[height=75mm]{Figure/direction_vector.eps}
   \caption{方向ベクトルの例}
   \label{fig:3-f2}
\end{figure}

次に、T字管と曲管の位置を示す並進ベクトルを用いて、両者を結ぶ線分のベクトルを式(\ref{eq:3-2})のように計算される。
\begin{equation}
\mathbf{l} = \mathbf{t}_C - \mathbf{t}_T
\label{eq:3-2}
\end{equation}

ここで、$\mathbf{l}$ はT字管から曲管への向きを示すベクトルである。
次に、この線分 $\mathbf{l}$ と各方向ベクトルの間の角度を計算する。ベクトル $\mathbf{a}$ と $\mathbf{b}$ の間の角度 $\theta$ は式(\ref{eq:3-3})で表される。

\begin{equation}
\cos \theta = \frac{\mathbf{a} \cdot \mathbf{b}}{\|\mathbf{a}\| \|\mathbf{b}\|}
\label{eq:3-3}
\end{equation}

$\|\mathbf{a}\|$ はベクトル $\mathbf{a}$ のノルムである。この式を用いて以下の角度を計算する。

\begin{equation}
\theta_{Tf} = \arccos \left( \frac{\mathbf{l} \cdot \mathbf{d}_{T,f}}{\|\mathbf{l}\| \|\mathbf{d}_{T,f}\|} \right)
\label{eq:3-4}
\end{equation}
\begin{equation}
\theta_{Cf} = \arccos \left( \frac{-\mathbf{l} \cdot \mathbf{d}_{C,f}}{\|\mathbf{l}\| \|\mathbf{d}_{C,f}\|} \right)
\label{eq:3-5}
\end{equation}

ここで、$-\mathbf{l}$ を用いるのは、曲管が線分の逆方向を向いている場合を考慮するためである。
最後に、これらの角度が閾値 $\theta_{th}$ よりも小さい場合、T字管と曲管は互いに向き合っていると判定する。具体的には以下の条件を満たす場合である。

\[
\theta_{Tf} < \theta_{th} \quad \text{かつ} \quad \theta_{Cf} < \theta_{th}
\]

\begin{figure}[htbt]
  \centering
   \includegraphics[height=55mm]{Figure/direction_th.eps}
   \caption{配管ペアマッチングの例}
   \label{fig:3-f3}
\end{figure}
図\ref{fig:3-f3}では、左に6D姿勢推定が大きくずれた場合の例を示し、右の図は6D姿勢推定が正確な場合の例を示している。
6D姿勢推定に大きな誤差があると、計算される角度も誤差が発生し、閾値 $\theta_{th}$ を超えて $\theta_{Tf}$ および $\theta_{Cf}$ が大きくなる。
その結果、T字管と曲管が正しく向き合っていないと判定される。
今回の配管設備は小規模であるため、角度差が大きくても閾値を緩和することで対応できる。
しかし、設備が大規模になると接続部の密度が増し、近接する異なる接続部を誤ってペアとして認識するリスクが高まる。
したがって、6D姿勢推定モデルの精度は、アイソメ図生成において極めて重要な要素となる。

\section{配管接続構造の探索}
配管設備の全体の接続関係を探索するために、深さ優先探索(Depth First Search, DFS)を利用する。
本手法では、配管の接続情報をグラフ構造として表現し、再帰的な探索を行うことで全ての接続関係を明らかにすることが可能である。\\
 まず、配管の接続点をノード、配管同士の接続をエッジとして捉え、ネットワーク全体をグラフとして表現する。
DFSにおいては、探索の起点を決定し、そこから接続された配管を順次たどり、探索中に訪問した配管を記録し、同じ配管を再度探索することがないよう管理する。
アルゴリズムの具体的な流れとしては、まず訪問済みの配管を記録する集合を空の状態で初期化し、探索を開始する。
次に、現在位置の配管を訪問済みとして記録し、その配管に接続されている他の配管を順に調べる、未訪問の配管が見つかった場合、その配管へ移動する。
これらの流れを再帰的に同様の手順を繰り返し、接続の確認が全て完了すると、探索した経路を逆戻りしながら他の接続を調べる。
また、ネットワーク全体を漏れなく探索するために、すべての配管に対してこの手法を適用する。
ただし、既に訪問済みの配管はスキップすることで、探索効率を向上させることができる。\\
 以上のDFSを用いた手法を取り入れることで、複雑に絡み合った配管ネットワークにおいても、すべての接続関係を網羅的に把握することが可能となる。

\section{配管情報の描画}
DFSによって配管の接続関係を把握した後、アイソメ図における配管間の距離を計算し、配管間を結ぶ線分を描画する。
配管間の距離は、各配管の並進ベクトル同士の3次元空間上の直線距離を計算し、ミリメートル(mm)単位で計算する。
この距離は、配管 $P_i$ および $P_j$ の並進ベクトルをそれぞれ $\mathbf{t}_i = (x_i, y_i, z_i)$ および $\mathbf{t}_j = (x_j, y_j, z_j)$ とすると、式(\ref{eq:3-6})で表される。

\begin{equation}
d_{ij} = \sqrt{(x_j - x_i)^2 + (y_j - y_i)^2 + (z_j - z_i)^2}
\label{eq:3-6}
\end{equation}

距離 $d_{ij}$ は配管間の直線距離を表しており、計算結果を図面上に表示する。
配管間を結ぶ線分は、配管 $P_i$ と $P_j$ の並進ベクトル $\mathbf{t}_i = (x_i, y_i, z_i)$ と $\mathbf{t}_j = (x_j, y_j, z_j)$ を始点と終点とし、直線で結ぶことで描画する。
もし配管に対応するペアが無い場合は、地面に接続されるものと仮定し、適切な長さの線分を描画する。\\
 配管設計図はPythonの\texttt{ezdxf}ライブラリを用いて生成することができる。
このライブラリを利用することで、線分の座標や角度を指定し、Drawing Exchange Format (DXF) 形式のファイルとして保存できる。
